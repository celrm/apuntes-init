% Idioma: español
\usepackage[spanish,es-lcroman]{babel} % Características del idioma
\usepackage[utf8]{inputenc} % Acentos

% Aspecto de página
%\usepackage{fancyhdr} % Headers
\usepackage{parskip} % Espacio entre párrafos
\usepackage{microtype} % Kerning y mejoras tipográficas
\usepackage{pagecolor} % Color de fondo
\usepackage{everypage} % \AddThispageHook {\tikz[remember picture,overlay] ... }

% Columnas y tablas
\usepackage{multicol}
\usepackage{multirow}
\usepackage{tabularx}
\usepackage{tabu}
\usepackage{diagbox} % Dividir una celda de un tabular
\def\arraystretch{1}

% Imágenes
\usepackage{graphicx}
\usepackage{wrapfig}
\usepackage{caption}
\graphicspath{{img/}}

% Símbolos
\usepackage{mathtools} % \text, \overset, \underset
\usepackage{amssymb} % \mathbb
\usepackage{upgreek} % \uptheta
\usepackage{amsthm} % QED
\usepackage{marvosym} % \EUR

% Formato
\usepackage{xcolor}
\usepackage{soul} % \so, \st, \hl
\usepackage{cancel} % \cancel, \bcancel, \xcancel, \cancelto
\newcommand*\canceling[2][thin]{\tikz[baseline] \node [strike out,draw,anchor=text,inner sep=0pt,text=black,#1]{#2};}

% Listas
\usepackage{enumerate} % Numeración de listas
\usepackage{outlines} % \0 \1 \2 \3 \4

% Dibujos y gráficas
\usepackage{tikz} 
\usepackage{pgfplots}
\usetikzlibrary{shapes,arrows,decorations.markings,shapes.misc,shapes.geometric,calc,pgfplots.groupplots}
\pgfplotsset{compat=1.13}
\usetikzlibrary{arrows,decorations.markings}

% Índices y títulos
\addto\captionsspanish{\renewcommand{\contentsname}{Índice}}
%\usepackage{titlesec} % Formato de títulos
%\usepackage{minitoc} % Índices de secciones
\usepackage{hyperref} % Links
\hypersetup{colorlinks}

% Otros
\usepackage[spanish,colorinlistoftodos,obeyDraft]{todonotes} % ToDo
\usepackage{pdfpages} % Incluir PDFs
\usepackage{qrcode} % QR
\usepackage{chemfig} % Química
\usepackage{wasysym} % Notas musicales

\usepackage[acronym]{glossaries} % Siglas y glosario
\usepackage{acronym} % Acrónimos

\usepackage{calc}

% Código
\usepackage{listingsutf8}
\lstset{
	inputencoding=utf8,
	extendedchars=true,
	literate=%
	{á}{{\'a}}1
	{é}{{\'e}}1
	{í}{{\'i}}1
	{ó}{{\'o}}1
	{ú}{{\'u}}1
	{Á}{{\'A}}1
	{É}{{\'E}}1
	{Í}{{\'I}}1
	{Ó}{{\'O}}1
	{Ú}{{\'U}}1
	{ñ}{{\~n}}1
	{Ñ}{{\~N}}1
	{¿}{{>}}1
	{¡}{{<}}1~
}

% Apéndices
\newcommand{\apendices}
{\section*{Apéndices}%
	\addcontentsline{toc}{section}{Apéndices}
	\setcounter{section}{0}%
	\setcounter{subsection}{0}%
	\renewcommand\thesection{\Alph{section}}%
}

\newenvironment{changemargin}[2]{%
	\begin{list}{}{%
			\setlength{\topsep}{0pt}%
			\setlength{\leftmargin}{#1}%
			\setlength{\rightmargin}{#2}%
			\setlength{\listparindent}{\parindent}%
			\setlength{\itemindent}{\parindent}%
			\setlength{\parsep}{\parskip}%
		}%
		\item[]
}
{\end{list}}

\title{\vfill\curso.-- \asignatura\vfill}
\subtitle{
	Profesor: \profesor\\\bigskip
	Curso: \annos\\\bigskip
	Grado en \carrera\\\bigskip
	Universidad Complutense de Madrid\vfill
}

\author{
	\textsf{
		\textbf{Celia Rubio Madrigal}
	}\\
	\textsf{
		\normalsize \textbf{Doble Grado en Ingeniería Informática y Matemáticas}
	}
}
\date{}

\usepackage[top=4.4cm, bottom=4.21cm]{geometry} % Márgenes


\usepackage[headsepline]{scrlayer-scrpage}
\clearpairofpagestyles
\ohead{\textsf{{Celia Rubio Madrigal}}}
\cfoot{\pagemark}
\ihead{\textsf{{Álgebra Lineal}}}

\renewcommand*\pagemark{{\usekomafont{pagenumber}\thepage}}
\addtokomafont{pageheadfoot}{\upshape}


\usepackage{tocbibind}
\usepackage{tocloft}
\usepackage{xpatch}

\newcommand{\mat}[3][]{\text{mat}^\mathcal{#2}_\mathcal{#1}\left(#3\right)}
\newcommand{\matb}[2]{\text{mat}_\mathcal{#1}\left(#2\right)}

\newlength{\myl}

\newcounter{dfs}
\newcommand{\listdfname}{Lista de definiciones}
\newlistof{definiciones}{todf}{\listdfname}
\xpretocmd{\listofdefiniciones}{\addcontentsline{toc}{section}{\listdfname}}{}{}
\newcommand{\df}[1]{\bigskip\phantomsection
	\stepcounter{dfs}
	\addcontentsline{todf}{definiciones}{\textsf{\textbf{\protect\numberline{\thedfs.} #1}}}
	\settowidth{\myl}{\textsf{\textbf{\underline{Def:} }}}
	\hspace*{-\myl}\textsf{\textbf{\underline{Def:} #1}}\medskip\\
}

\usepackage{setspace}

\newcommand{\listtrname}{Lista de teoremas, lemas y corolarios}
\newlistof{teoremas}{totr}{\listtrname}
\xpretocmd{\listofteoremas}{\addcontentsline{toc}{section}{\listtrname}}{}{}
\newcommand{\tr}[2]{\bigskip\phantomsection
	\label{t:#1}
	\addcontentsline{totr}{teoremas}{\textsf{\textbf{\protect\numberline{\ifx&#1&---\else#1\fi.}\ \ #2}}}
	\settowidth{\myl}{\textsf{\textbf{\underline{T.#1:} }}}
	\hspace*{-\myl}\textsf{\textbf{\underline{T.#1:} #2}}\medskip\\
}

\newcommand{\lm}[1]{\bigskip\phantomsection
	\addcontentsline{totr}{teoremas}{\textsf{\textbf{\protect\numberline{---.}\ \ #1}}}
	\settowidth{\myl}{\textsf{\textbf{\underline{Lema:} }}}
	\hspace*{-\myl}\textsf{\textbf{\underline{Lema:} #1}}\medskip\\
}

\newcommand{\cor}[1]{\bigskip\phantomsection
	\addcontentsline{totr}{teoremas}{\textsf{\textbf{\protect\numberline{---.}\ \ #1}}}
	\settowidth{\myl}{\textsf{\textbf{\underline{Corolario:} }}}
	\hspace*{-\myl}\textsf{\textbf{\underline{Corolario:} #1}}\medskip\\
}

\newcommand{\ob}[1]{\bigskip\phantomsection
	\settowidth{\myl}{\textsf{\textbf{\underline{Obs:} }}}
	\hspace*{-\myl}\textsf{\textbf{\underline{Obs:} #1}}\medskip
}

\newcommand{\ar}{\rotatebox[origin=c]{180}{\LARGE\ensuremath{\Lsh}}}